\section{Introdu��o}

A Computacao Ub�qua, ou ubicomp, vem sendo tema de diversas pesquisas desde o
in�cio dos anos 90. Seu enfoque est� em que a computa��o deveria ser algo
invis�vel, nos servindo e exigindo o m�nimo de esforco poss�vel~\cite{weiser1,
weiser2}. Permitindo assim, que o usu�rio tenha mais foco na tarefa em execu��o
que na ferramenta. Um ambiente com tais caracter�sticas � chamado de inteligente
(\textit{smart space}) pois busca auxiliar seus usu�rios de modo proativo e
transparente.

A intelig�ncia esperada nestes ambientes � fruto de aplica��es que tratam os
dados fornecidos pelos diversos dispositivos presentes, bem como coordenam suas
a��es. Para que tudo isto ocorra de forma transparente estes softwares devem ser
sens�veis ao contexto em que est�o presentes. Tendo em vista a dinamicidade do
\textit{smart space}, uma vasta gama de informa��es podem ser utilizadas para se
construir este contexto. De maneira especial podemos olhar para as informa��es
de identifica��o e localiza��o do usu�rio. De posse destas informa��es �
poss�vel personalizar as a��es bem como direcion�-las aos locais corretos de
atua��o.

Apresentaremos aqui um sistema de rastreamento, localizacao e identificacao de
usu�rios em um ambiente inteligente. Este foi denominado TRUE, um acr�nimo para
\textit{Tracking and Recognizing User in the Environment}. O Sistema TRUE obt�m
informa��es do ambiente utilizando o sensor Kinect~\cite{kinect_microsoft} da
Microsoft e disponibiliza seus resultados para as aplica��es atrav�s do
Middleware uOS~\cite{fabriciobuzzeto2, fabriciobuzzeto}.